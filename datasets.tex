\newpage
 \section{XLCoST: Cross-Language Code Snippet Transfer}

\subsection{Структура и особенности}
XLCoST (Cross-Language Code Snippet Transfer) — это мультиязычный датасет, разработанный для задач трансляции кода между языками программирования и генерации кода из текстовых описаний. Он содержит парные данные для 8 языков: Python, Java, C++, C\#, JavaScript, PHP, Go и Ruby. Каждая запись включает:

    
- Исходный код на одном языке.
    
- Соответствующий перевод на другой язык.
    
- Текстовое описание функционала на английском языке.


Датасет разделен на три подмножества:
\begin{enumerate}
    \item \textbf{Code-to-Code}: Пары кода на разных языках (например, Java ↔ C++).
    \item \textbf{Text-to-Code}: Описания на естественном языке и соответствующий код.
    \item \textbf{Documentation}: Расширенные комментарии и документация.
\end{enumerate}

Общий объем данных превышает 1.2 миллиона примеров, собранных из открытых репозиториев GitHub и Stack Overflow.

\subsection{Применение и исследования}
XLCoST используется для обучения моделей, способных выполнять:

    
- Трансляцию кода между языками (например, автоматический перенос алгоритма с Python на Java).
    
- Генерацию кода из текстовых спецификаций.
    
- Синхронизацию документации при изменении кодовой базы.


Особенность датасета — акцент на параллельность данных, что позволяет исследовать кросс-языковые зависимости. Например, в работе Ming Zhu et al. (2022) модель на основе XLCoST демонстрирует точность 78\% в задачах перевода между Java и Python.

Датасет активно применяется в исследованиях мультиязычных моделей, таких как CodeBERT и PLBART, а также в коммерческих инструментах рефакторинга.

\subsubsection*{Ссылки}
\href{https://arxiv.org/abs/2203.04225}{Zhu, M., et al. "XLCoST: A Benchmark Dataset for Cross-Language Code Snippet Transfer." arXiv:2203.04225 (2022)}.

\newpage

\section{CodeSearchNet: Семантический поиск кода}

\subsection{Структура и языки}
CodeSearchNet (CSN) — датасет, разработанный GitHub для обучения моделей семантического поиска кода. Он охватывает 6 языков: Python, JavaScript, Ruby, Go, Java, PHP. Каждая запись содержит:

    
- Фрагмент кода (функцию или метод).
    
- Текстовое описание его функционала (на английском языке).
    
- Метаданные (репозиторий, лицензия, звезды GitHub).


Объем данных — 2.3 миллиона пар код-описание, что делает CSN одним из крупнейших ресурсов для NLP-задач, связанных с кодом. Данные собраны из публичных репозиториев с лицензиями MIT, Apache 2.0 и GPL.

\subsection{Практическое использование}
CodeSearchNet решает две ключевые задачи:
\begin{enumerate}
    \item Поиск кода по текстовому запросу (например, "сортировка списка по убыванию").
    \item Генерация описаний для существующего кода.
\end{enumerate}

Датасет стал основой для моделей вроде CodeBERT и UniXcoder, которые используются в GitHub Copilot для предложения релевантных фрагментов кода. В исследовании Husain et al. (2019) модель на CSN достигла точности 72\% в поиске кода для Python.

\subsubsection*{Ссылки}
\href{https://arxiv.org/abs/1909.09436}{Husain, H., et al. "CodeSearchNet Challenge: Evaluating the State of Semantic Code Search." arXiv:1909.09436 (2019)}.

\newpage

\section{CodeXGLUE: Бенчмарк для оценки моделей}

\subsection{Архитектура и задачи}
CodeXGLUE (Code eXamination General Language Understanding Evaluation) — комплексный бенчмарк от Microsoft, включающий 11 задач для оценки моделей обработки кода:

    
- Code Completion (автодополнение).
    
- Code Repair (исправление ошибок).
    
- Text-to-Code Generation (генерация кода из текста).
    
- Code Translation (перевод между языками).


Датасет поддерживает языки: Python, Java, C\#, JavaScript и PHP. Его структура объединяет несколько существующих ресурсов (например, CodeSearchNet) и добавляет новые, такие как Code2Seq для генерации последовательностей.

\subsection{Роль в исследованиях}
CodeXGLUE стандартизирует оценку моделей, таких как Codex (OpenAI) и GraphCodeBERT, позволяя сравнивать их эффективность. Например, в задаче исправления ошибок модель Codex достигает точности 64\%, тогда как специализированные модели (например, DeepDebug) показывают 71\% (Lu et al., 2021).

Датасет также включает метрики оценки (BLEU, Accuracy, F1) и лидерборды, что стимулирует конкуренцию в научном сообществе.

\subsubsection*{Ссылки}
\href{https://dl.acm.org/doi/10.1145/3484577}{Lu, S., et al. "CodeXGLUE: A Benchmark Dataset for Code Intelligence." ACM Transactions on Software Engineering and Methodology (2021)}.

\newpage

\section{Сравнение}
Все три датасета решают взаимодополняющие задачи:

    
- \textbf{XLCoST} фокусируется на мультиязычности и трансляции кода.
    
- \textbf{CodeSearchNet} оптимизирован для семантического поиска.
    
- \textbf{CodeXGLUE} обеспечивает стандартизацию оценки моделей.


Их объединяет использование данных из открытых источников (GitHub, Stack Overflow) и поддержка популярных языков (Python, Java). Однако XLCoST выделяется включением C++ и Ruby, а CodeXGLUE — разнообразием задач.

Эти датасеты стали основой для прорывов в генерации кода, например, в GitHub Copilot и Amazon CodeWhisperer. Дальнейшее развитие области связано с увеличением объема данных и улучшением обработки низкоресурсных языков (например, Kotlin).

\textbf{Перспективы:}

    
- Интеграция датасетов для создания универсальных моделей.
    
- Расширение поддержки языков для нишевых экосистем (Rust, Swift).
    
- Применение в образовании (автоматическая проверка заданий).


Таким образом, XLCoST, CodeSearchNet и CodeXGLUE играют ключевую роль в эволюции инструментов разработки и методов машинного обучения, связанных с кодом.


