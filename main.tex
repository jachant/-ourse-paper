\documentclass[14pt]{article}
% \usepackage[14pt]{extsizes}
\usepackage[utf8]{inputenc}
\usepackage{amssymb,amsthm}
\usepackage{amsmath}
\theoremstyle{definition}
\newtheorem{definition}{Определение}
\usepackage[russian]{babel}
\usepackage[shortlabels]{enumitem}
\newtheorem{theorem}{\bf Теорема}
\usepackage{graphicx}
\usepackage[left=3cm, right=3cm, top=3cm, bottom=3cm]{geometry}
\setlength{\parindent}{0cm}
\newenvironment{ourproof}{\\ \textit{Доказательство.}\\ }{$\hfill \heartsuit$}
\usepackage{ amssymb }
\usepackage{ dsfont }
\newtheorem{exercise}{Упражнение}
\newtheorem{example}{Пример}
\setlength{\parindent}{5ex}
\newtheorem{rem}{Замечание}[section]
\newtheorem{proposition}{Предложение}[section]
\usepackage[T1]{fontenc}
\usepackage{ mathrsfs }
\usepackage{mathrsfs}
\usepackage{ upgreek }
\usepackage{wrapfig}
\usepackage{textcomp}

\linespread{1.5} 
\frenchspacing

    

\usepackage{xcolor}
\usepackage{hyperref}


\begin{document}


\begin{titlepage}
  \begin{center}
    \normalsize
   \textbf {Правительство Российской Федерации\\ 
Федеральное государственное автономное образовательное учреждение\\
   высшего профессионального образования\\
    «Национальный исследовательский университет» \\
     «Высшая школа экономики»}
   

    
    
    
    \textbf {Нижегородский филиал}
    
  \vfill
    Факультет математики, информатики и компьютерных наук\\
    
    
   
    \vfill

    \textbf{ КУРСОВАЯ РАБОТА}\\[5mm]
    
    {\normalsize  \textbf{О кодовых LLM и способах оценки качества моделей для класса задач
суммаризации кода}}
    
  \bigskip
    
    
\end{center}
\vfill

\newlength{\ML}
\settowidth{\ML}{«\underline{\hspace{0.7cm}}» 
\underline{\hspace{2cm}}}
\hfill
\begin{minipage}{0.5\textwidth}
  Выполнил:\\
 Студент 2 курса группы 23КНТ6    
   \\
 Антонов Артём Владимирович\\ 


 \\Научный руководитель:\\
 Доцент кафедры\\ 
 фундаментальной математики\\
Казаков Алексей  Олегович\\
  \vspace{1cm}
 {\hspace{2.5cm}}
\end{minipage}%
\vfill

\begin{center}
  Нижний Новгород\\Май 2025 г.
\end{center}

\end{titlepage}

\pagebreake[2]


\newpage
\tableofcontents
\newpage
\section{Введение}
В современном мире большие языковые модели стали неотъемлемой частью повседневной жизни. Они используются в различных сферах деятельности, таких как образование, бизнес, наука и технологии. В этой работе мы рассмотрим основные аспекты применения больших языковых моделей и их влияние на нашу жизнь.

Большие языковые модели представляют собой мощные инструменты, способные обрабатывать огромные объёмы текста и генерировать тексты, которые могут быть полезны для решения различных задач. Они основаны на глубоком обучении и используют огромные объёмы данных для обучения.

Одним из способов повышения эффективности работы с большими языковыми моделями является обобщение кода. Обобщение кода — это процесс преобразования исходного кода в более общий или абстрактный вид, который может быть использован для решения множества задач. Этот метод позволяет разработчикам создавать более универсальные и гибкие программы, которые могут быть адаптированы к различным контекстам и требованиям.

Обобщение кода (преобразование кода в текст): Целью задачи обобщения кода является генерация
описаний кода на естественном языке, который предоставляется в качестве входных данных. Мы выполняем эту задачу
двумя способами: генерируем сводку на уровне фрагмента, используя пары комментарий-фрагмент, и генерируем
сводку на уровне проблемы, используя пары описания проблемы и программного кода. Приложения
этой задачи включают повышение понятности раскомментированного или незнакомого кода для начинающих
пользователей и упрощение совместной работы, а также обучения.


//todo: definition 

We use the following metrics to evaluate different tasks proposed in this work: (i) BLEU (Papineni
et al., 2002) score to evaluate code-to-text generation tasks;, (ii) BLEU and CodeBLEU4 (Ren et al.,
2020) to evaluate code-to-code and text-to code generation tasks, and (iii) Mean Reciprocal Rank
(MRR) to evaluate retrieval tasks.

Задача исследования также включает сравнение различных датасетов и выявление причин, по которым два похожих датасета могут иметь разную популярность использования.

 
 
 \section{part 1}
 


\newpage
 \section{part 2}
 
 
 




\newpage
\section{Заключение}





\newpage
\addcontentsline{toc}{section}{\bf{Список литературы}}
\begin{thebibliography}{99}

\bibitem{RuelTakens}Рюэль Д., Такенс Ф., О природе турбулентности. // в кн. “Странные аттракторы”, М.: Мир, 1981.

\bibitem{Lorenz1} Lorenz E.N., Deterministic nonperiodic flow. // J. of the Atmospheric Sciences, 1963, 20, 130-141 [Перевод на русский язык: Эдвард Н. Лоренц Детерминированное непериодическое течение – в кн. “Странные аттракторы”, М.: Мир, 1981.


\bibitem{SH3} Шильников Л.П., Теория бифуркаций и модель Лоренца.// Дополнение I к кни- ге Дж.Марсдена и М.Мак- Кракена “Бифуркация рождения цикла и ее приложения.” М., Мир, 1980.

\bibitem{ABSH2} Афраймович В.С., Быков В.В., Шильников Л.П., О притягивающих негрубых мно- жествах типа аттрактора Лоренца.// Труды ММО, 1982.

\bibitem{ABSH1} Афраймович В.С., Быков В.В., Шильников Л.П. О возникновении и структуре ат- трактора Лоренца.// ДАН СССР, 1977.

\bibitem{ShaShi} M. V. Shashkov, L. P. Shilnikov, “On the existence of a smooth invariant foliation in Lorenz-type mappings”, Differ. Uravn. (1994)

\bibitem{GonKazTur} Sergey Gonchenko , Alexey Kazakov,  Dmitry Turaev Wild pseudohyperbolic attractor in a four-dimensional Lorenz system. 2018

\bibitem{MalSaf} M. Malkin, K. Safonov Entropy charts and bifurcations for Lorenz maps with infinite derivatives.2021


\end{thebibliography}

\end{document}
